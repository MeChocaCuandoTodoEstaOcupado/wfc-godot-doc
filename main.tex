\documentclass[12pt,letterpaper]{book}

% =================================================================
% CONFIGURACION BaSICA
% =================================================================

% Idioma y Codificación
\usepackage[utf8]{inputenc}
\usepackage[spanish,es-tabla]{babel}

% Márgenes
\usepackage[left=3cm, right=2cm, top=2cm, bottom=2cm]{geometry}

% Fuente (Arial-like)
\renewcommand{\familydefault}{\sfdefault}

% Espaciado
\usepackage{setspace} % Para control de interlineado 
\usepackage{parskip} % Elimina sangría y usa espacio entre párrafos.

% Matematicas
\usepackage{amsmath, amsthm, amssymb}

% =================================================================
% FORMATO Y ESTILO 
% =================================================================

\usepackage{xcolor} % Necesario para definir colores

% Hipervínculos 
\usepackage{hyperref} 
% Aquí se asegura que las referencias sean negras y sin recuadros de color.
\hypersetup{
	colorlinks=true,
	linkcolor=black,    % Color de los enlaces internos (secciones, páginas)
	citecolor=black,    % Color de las citas bibliográficas
	urlcolor=black,     % Color de los enlaces URL
}

% Encabezados y capítulos
\usepackage{fancyhdr}
\usepackage{titlesec}
\usepackage[Lenny]{fncychap}
\usepackage{emptypage} % Para páginas vacías sin encabezados ni numeración

% =================================================================
%FIGURAS Y TABLAS
% =================================================================

% Figuras
\usepackage{graphicx}
\usepackage{caption}
\usepackage{float}

% Tablas
\usepackage{tabularray}
\UseTblrLibrary{booktabs}
\UseTblrLibrary{siunitx}
\usepackage{array}
\usepackage{multirow}
\usepackage{makecell} 

% Bibliografía
\usepackage[round,authoryear]{natbib}
\usepackage{appendix} 


% =================================================================
% INDICES
% =================================================================

%\usepackage{textcase} % permite usar el comando seguro \MakeUppercase en lugar del comando de bajo nivel \uppercase
\usepackage{tocloft}
%%%%%%%%%%%%%%%%%%%%%%%%%%%%%%%%%% aqui comienza el documento %%%%%%%%%%%%%%%%%%%%%%%%%
\begin{document}
	
%formato del texto "capitulo"
\titleformat{\chapter}[display]
{\fontsize{14}{16}\Large\filcenter} 
{{\chaptertitlename} \Roman{chapter}} %  
{20pt}  % Espacio entre "Capítulo I" y el título
{\bfseries\MakeUppercase} 

% Formato de títulos y subtítulos
\titleformat{\section}
  {\fontsize{12}{14}\bfseries\uppercase} % Estilo del título de sección
  {\thesection} % Etiqueta del capítulo
  {1em} % Espacio horizontal entre la etiqueta y el título
  {} % Código previo al título de sección

\titleformat{\subsection}{\bfseries}{\thesubsection}{1em}{}
\titleformat{\subsubsection}{\bfseries}{\thesubsubsection}{1em}{}

%formato indice
\renewcommand{\cftchapfont}{\bfseries\uppercase} %capitulos
\renewcommand{\cftsecfont}{\uppercase} % Secciones


% agrega puntos ... a los capitulos preliminares en el indice
%\renewcommand{\cftchapdotsep}{\cftdotsep}

% Formato de párrafo
\setlength{\parindent}{0pt} % Sin sangría
\setlength{\parskip}{6pt} % Espaciado entre párrafos

% Configuración de página
\pagestyle{fancy} % Estilo de página fancy
\fancyhf{} % Borra todos los encabezados y pies de página actuales
\fancyfoot[C]{\thepage} % Número de página centrado en el pie de página
\renewcommand{\headrulewidth}{0pt} % Sin línea de encabezado
\renewcommand{\footrulewidth}{0pt} % Sin línea de pie de página


% Formato de figuras
\captionsetup{justification=centering, font={footnotesize, sf}, labelfont={bf, sf}}


% Configuracion del caption
\captionsetup[table]{skip=4mm}

	
	% portada
	\begin{titlepage}
    \begin{minipage}{2.16cm}
        \begin{center}
            \includegraphics[width=2.16cm,height=3.2cm]{img/logoUMSS.png}
        \end{center}
    \end{minipage}
    \hfill
    \begin{minipage}{10cm}
        \begin{center}
            \large{ \textbf{\MakeUppercase{Universidad mayor de san simón}} }\\
            \normalsize{ \textbf{\MakeUppercase{Facultad de ciencias y tecnologia}} }\\
            \small{ \textbf{\MakeUppercase{Ingeniería informática}} }
        \end{center}
    \end{minipage}
    \hfill
    \begin{minipage}{2.55cm}
        \begin{center}
            \includegraphics[width=2.55cm,height=2.55cm]{img/logoFacultad.jpg}
        \end{center}
    \end{minipage}
    \vspace{5cm}\\

    \begin{center}
        \textbf{\Large\MakeUppercase{Desarrollar un Asset para facilitar la creación procedural de mapas tipo mosaico 2D y 3D a desarrolladores de videojuegos.}}
    \end{center}

    \vspace{4cm}
    \begin{center}
        Proyecto de Grado Presentado para optar al Diploma Académico de Licenciatura en Ingeniería Informática
    \end{center}

    \vspace{2cm}
    \textnormal{\textbf{Presentado por:} Ríos Cardozo Nicolás Luis}\\
    \textnormal{\textbf{Tutor:} Lic. -}\\

    \vspace{2.5cm}
    \begin{center}
        \large\textbf{\MakeUppercase{Cochabamba - Bolivia}}\\
        {II, 2025}\\
    \end{center}
\end{titlepage}
	
	% indices
	\frontmatter
	\include{preliminares/dedicatoria}
	\include{preliminares/agradecimientos}
	\include{preliminares/fichaResumen}
	\tableofcontents
	\clearpage
	\listoffigures
	\clearpage
	\listoftables
	
	\mainmatter
	    
\chapter{Introducción}

El presente proyecto consiste en resolver la necesidad de herramientas mas accesibles en la generación procedural de mapas en el desarrollo de videojuegos
\\\\ %esto es salto de linea

%------------------------------------------------------------%
\section{Antecedentes}

El desarrollo de videojuegos es un área en crecimiento mas accesible de ingresar gracias a la facilidad que generan el uso de motores de juegos, y herramientas para estos generados por sus respectivas comunidades.

A pesar de todos los beneficios de usar un motor de videojuegos como base para el desarrollo, hacer videojuegos es actualmente una tarea que demanda de muchos aspectos en los que trabajar por lo que se puede terminar tomando mucho tiempo en terminar de implementar todos los aspectos necesarios que lo involucra

Entre las necesidades mas comunes para un desarrollador de videojuegos es la creación de mapas que puede ser una tarea tardía, por lo que muchos proyecto pequeños optan por la generación automática de estos por diferentes métodos



\emph{cita ejemplo}
\citep{cappuchino}
\\\\

%------------------------------------------------------------%
\section{Descripción del problema}

Un videojuego es un proyecto pesado que incluye muchas áreas a desarrollar entre ellos diseñar el mapa de los niveles así que para aligerar la carga los desarrolladores pueden usar Assets.

Un Asset es como una biblioteca que puede incluir scripts de código, imágenes, modelos 3D/2D, etc disponibles a usar para un nuevo proyecto, estos suelen estar publicados para proyectos usando algún motor de videojuegos especifico para ser compatible.




%------------------------------------------------------------%

\subsection{Definición del Problema}
Dificultad para la generación de mapas en en videojuegos.

\begin{figure}[H]
    \centering
    \includegraphics[width=0.6\textwidth]{img/Screenshot_20240819_191212.png}
    \caption{Árbol de problemas, centrado el la generación de mapas para videojuegos.}
    \label{fig:arbol-de-problemas}
\end{figure}
%------------------------------------------------------------%
\section{Objetivos}
A continuación se presentan el objetivo general y los objetivos específicos en este proyecto de grado.  
\subsection{Objetivo general}
Desarrollar un Asset para facilitar la creación procedural de mapas tipo mosaico 2D y 3D a desarrolladores de videojuegos

 

\subsection{Objetivos específicos}
\begin{enumerate}
\item Investigar la técnica de wave function colapse para la generación de mapas

\item Investigar la técnica de seeding/semillas para añadir control y reproducibilidad de los mapas resultados generados

\item Investigar un motor de videojuegos para implementar un Asset compatible con esa tecnología

\item Implementar la funcionalidad de generación de mapas compatible con las herramientas del motor de videojuegos

\item Definir casos de uso para las pruebas de la funcionalidad

\item Hacer el Asset publico para cualquier desarrollador de videojuegos


\end{enumerate}
%------------------------------------------------------------%
\section{Justificación}

Como se menciono en puntos anteriores hacer videojuegos es un tarea pesada, entre las necesidades mas comunes para un desarrollador de videojuegos es la creación de mapas, para aligerar tal carga se busca publicar un Asset que ayude a desarrolladores a facilitar esa área del desarrollo de videojuegos.

%------------------------------------------------------------%
\section{Límites y alcances}
El presente trabajo de grado se enfoca en los siguientes aspectos:
\begin{itemize}
\item 
\item 
\item 
\item 
\end{itemize} 
  

%------------------------------------------------------------%
\section{Metodología de desarrollo}

Al momento de avanzar en el procedimiento se definió que el proceso a seguir necesitaría ser un proceso ágil para poder avanzar evitando interrupciones y acomodarse a necesidades surgientes en el desarrollo.
Entre los procesos ágiles conocidos se eligió kanban por el echo de poder amoldarse al trabajo en solitario requerido y ser flexible con flujo de trabajo para evitar posibles atascos.
Las tareas principales de investigación que se definió serían las siguientes:

\begin{table}[H]
    \centering
\caption{Cronograma de actividades}
\label{tab:actvd}
\begin{tabular}{|>{\centering\arraybackslash}p{0.14\linewidth}|>{\centering\arraybackslash}p{0.24\linewidth}|>{\centering\arraybackslash}p{0.24\linewidth}|>{\centering\arraybackslash}p{0.24\linewidth}|}
\hline 
    
        Nro. Objetivo Específico & Actividades & Recursos Necesarios &Resultados a obtener \\ \hline 
    
         1& Investigación de la implementación de las técnicas wave function collapse y seeding &  & Diseño de algoritmos a usar\\ \hline 
    
         2& Investigación del motor a usar y herramientas para generar mapas & lenguaje de programación compatible con el motor & Diseño de uso en el motor y herramientas a usar de este\\ \hline 
         
         3& Diseñar y desarrollar biblioteca de generación de mapas & lenguaje de programación compatible con el motor &biblioteca de generación de mapas \\ \hline
    
         4& Desarrollar pruebas para la biblioteca creada & Motor y biblioteca de unit test & conjunto de test de unidad \\ \hline 
    
         5& publicar Asset implementado & Motor, biblioteca de generación de mapas & Asset \\ \hline 
    
    

    \end{tabular}
\end{table}

    Gran parte de los puntos no requieren la totalidad de la investigación de puntos anteriores así que se irán generando tareas a cumplir según lo de lo que ya no se tenga bloqueos para su desarrollo.
	
\chapter{MARCO TEÓRICO}

En este capítulo se explican los conceptos de los mapas estilo mosaico y las técnicas a usar para poder generarlos procedural-mente, también se profundizara en la tecnología de los motores de videojuegos a usar
 \footnote{métodos adaptativos: métodos basados en la estimación del error } 
.


\section{MAPAS ESTILO MOSAICO O TILEMAPS}

Los mapas estilo mosaico o también llamados tilemaps son una técnica común en el desarrollo de videojuegos especialmente en juegos 2D, que consiste en construir el mapa del mundo o nivel de juego a base de pequeñas imágenes con forma usualmente cuadrada a los que se les llaman mosaicos o tiles,
los beneficios de usar esto es que no necesitar grandes imágenes que pueden pesar mucho en cambio se construye usando pequeñas imágenes que se pueden repetir varias vecen el diferentes partes del mapa.

Otro beneficio de esto es que se pueden poner en matrices de 2 y 3 dimensiones lo que hace sencillo poder definir las posiciones de los tiles usando algoritmos.

En el caso de mapas 3D se usan modelos 3D  o también llamados 3D mesh, en vez de imágenes  obteniendo los mismos beneficios.


\section{WAVE FUNCTION COLLAPSE}

El colapso de la función de onda o wave function collapse es un algoritmo basado en un concepto de física cuántica con el mismo nombre. Cuando se mide un sistema cuando en superposición de estados su función de onda se colapsa a un solo estado, por lo que se considera el estado es indeterminado hasta que se haga una medición, experimentos conocidos de este fenómeno son el gato de Schrödinger o el experimento de la doble rendija.

\begin{figure}[H]
	\centering
	\includegraphics[width=0.6\textwidth]{img/screenshot-2025-10-26_10-24-06.png}
	\caption{experimento de la doble rendija.}
	\label{fig:doble-rendija}
\end{figure}

\subsection{Algoritmos genéticos paralelos y distribuidos (PGA , DGA)}
\subsection{Algoritmos genéticos híbridos (HGA)}
\subsection{Algoritmos genéticos adaptativos (AGA)}
\subsection{Algoritmos genéticos rápido desordenado (FmGA)}
\subsection{Algoritmos genéticos muestreo independiente (ISGA)}
\section{APLICACIONES ACTUALES EN BASE A ALGORITMOS GENÉTICOS}
El uso de algoritmos genéticos ha estado subiendo exponencialmente en los últimos años. Los algoritmos genéticos son usados en su mayoría para los problemas de optimización y la resolución de problemas NP difíciles.
Una de las áreas donde se puede ver a los algoritmos genéticos ser aplicado es el aprendizaje de máquinas, Velez-Langs Oswaldo  y Santos Carlos (2014)  usaron algoritmos genéticos para su implementación de un sistema recomendador, que demostró su correctitud en la recomendación de películas al usuario. Lars Bungum y Björn Gambäck (2010) demostraron que la programación evolutiva puede ser efectivamente usada en las diferentes etapas del procesamiento del lenguaje natural.

\section{MOTORES DE VIDEOJUEGOS}

Usar motores de videojuegos como base de un proyecto es actualmente lo mas frecuente en el desarrollo de videojuegos ya que bastante costoso implementar desde 0 todas las funciones que aportan estos.

Por lo que los Assets necesitan ser compatibles con el motor de videojuegos que se elija usar entre ellos las opciones mas conocidas y completas serian Unity y Unreal

Unity y Unreal son motores de videojuegos los cuales han estado dominando la industria por un buen tiempo por lo cual tienen un buen repertorio de Assets publicados dando varias opciones a nuevos desarrolladores, aparte de estos hay varios otros motores entre los cuales uno adquirió popularidad recientemente Godot.

Godot es un motor de videojuegos gratuito de código abierto creado originalmente en Argentina por Ariel Manzur y Juan Linietsky como un proyecto cerrado el cual pasaría a lanzarse como código abierto el 14 de enero de 2014 con licencia MIT
Aunque adquirió popularidad rápidamente el aporte de la comunidad es bastante pequeño comparado con otros motores que llevan mas tiempo encabezando el mercado.

Una gran ventaja de usar Godot por encima de motores como Unity y Unreal es que es completamente gratuito por lo que desarrolladores no necesitan pagar licencias para publicar sus juegos, también que al ser código abierto no esta sujeto a las políticas de ninguna compañía por lo que se puede tener mas libertad creativa.

Otra ventaja de Godot es que al poder usar el mismo lenguaje de programación C\# usado en Unity y Unreal estos proyectos serian mas fáciles de exportarse a este nuevo motor de ser necesario y viceversa.

\section{ESTADO DEL ARTE}
\section{TECNOLOGÍAS}
\subsection{Lenguaje de programación}

Para lograr el cometido se eligió el lenguaje de programación Java por múltiples motivos que veremos a continuación.\\\\
Java es un lenguaje popular que fue publicado en 1995 por Sun Microsystems, entre las ventajas que tiene es que es un lenguaje moderadamente fácil de entender y mantiene un alto rendimiento.
Al ser uno de los lenguajes más populares que existen java cuenta con una comunidad grande y extensa documentación disponible para usar, tambien le da el beneficio de tener una basta cantidad de frameworks y diferentes bibliotecas hechas para este lenguaje. de las cuales veremos a continuación las que se eligió para llevar a cabo este proyecto.

\subsection{Spring}

Spring es un framework para Java posiblemente el más popular en la actualidad que ofrece varias herramientas al desarrollador para poder simplificar bastante el trabajo en la creación de aplicaciones web.
Spring es bastante modular permitiendo solo importar las librerías relevantes para el trabajo.\\\\
En este proyecto se busca poder usar Spring para poder dar un API capaz de ser usado para poder dar los resultados a los estudiantes de los horarios que desean optimizar y también pueda dar un sistema con un mínimo aceptable de seguridad a los administradores para popular la información necesaria de los horarios.

\subsection{Jenetics}
Jenetics es una biblioteca para java que implementa varios de los conceptos de algoritmos evolutivos y utiliza la funcionalidad de Stream de java para la iteración de generaciones, lo que permite concentrarse en la resolución del problema de este proyecto que es la optimización de los horarios elegidos por estudiantes, que principalmente radica en la definición de la función fitness.

\subsection{Base de datos}
Para el almacenamiento de información necesaria en este proyecto se eligió utilizar mysql por su sencillez popularidad y estabilidad.

\subsection{Angular}
En cuanto la interfaz de usuario se utilizó Angular al ser un framework bastante completo y robusto, conteniendo todo lo necesario para las necesidades de este proyecto.
\section{PROCESO DE DESARROLLO}
Como se mencionó en el anterior capítulo este proyecto estará desarrollado usando el proceso ágil conocido como Kanban. en los siguientes puntos se profundizará más respecto al proceso y el motivo de su elección
\subsection{Kanban}
\subsection{Por qué no otros procesos}

	
\chapter{MARCO DE APLICACIÓN}
\section{Diseño}
diseño
\subsection{subsubtitulo}
subdiseño
\section{Implementación}
implementations
\subsection{Herramientas de software}
herramentations
	\chapter{DESARROLLO DEL PROYECTO}
\section{DISEÑO DE ARQUITECTURA}
Se opta por una arquitectura MVP(modelo vista presentador) para mantener desacoplada la interfaz de usuario y tener un API de uso genérico que puede ser reutilizado por otros proyectos.
Spring basa su arquitectura en MVC(Modelo vista controlador) y es la arquitectura que seguiría el componente de backend en este proyecto pero, al integrar también una interfaz de usuario ajena a Spring boot el controlador desde el punto de vista general del proyecto pasaría a usarse como un intermediario funcionando como un presentador que se encarga de la lógica y enviar o recibir información de la  vista a petición.
\begin{figure}[H]
    \centering
    \includegraphics[width=0.75\linewidth]{img/componentDiagram.png}
    \caption{Diagrama de componentes}
    \label{fig:enter-label}
\end{figure}
\section{DISEÑO DE ALGORITMO GENÉTICO}
Tras el análisis hecho a las variaciones mencionadas en el capítulo se decidió por usar la siguiente estructura [TODO].algoritmo genético simple.[TODO] analizar matemáticamente las probabilidades y cantidades máximas de combinaciones, permutaciones, relevancia en complejidad del problema, como son 6 max de materias tomadas posiblemente se utilice el algoritmo genético base pero cada parte también tiene formas de implementar revisar cuales estan disponible en Jenetics..[TODO].. dado ya el algoritmo a seguir, para el desarrollo de esto se utilizó como apoyo Jenetics una librería para el lenguaje de programación Java que cuenta con bases para la implementación de algoritmos genéticos que ayudan a agilizar la implementación de la aplicación
\section{DISEÑO DE BASE DE DATOS}
Para implementar un sistema capaz de recomendar horarios a los estudiantes, primero se necesita la información con la que trabajar, en vez de tediosamente re ingresar todos los datos para cada uso es más conveniente poder almacenar la información de los horarios en una base de datos a la cual el sistema tenga acceso para poder revisar en cada request de algún usuario para poder generar un horario óptimo.
Dado lo anterior dicho se trabajó en un diseño de la información a guardar respecto a los horarios que incluye tanto los tiempos de ingreso de las materias y de finalización, como también información básica de los docentes que darán tales materias y los grupos disponibles que estarán en la gestión.

\begin{figure}[H]
    \centering
    \includegraphics[width=0.5\linewidth]{img/db.png}
    \caption{Diagrama entidad relación}
    \label{fig:enter-label1}
\end{figure}

En la anterior imagen se puede visualizar la relación entre los datos necesario para definición de horarios

\section{DISEÑO DE INTERFAZ DE USUARIO}

En la interfaz de usuario se decidio optar por una vista simple que sea capaz de mostrar el horario resultado de las preferencias del estudiante y un menu para seleccion dichas preferecia.
\\\\
Un menu desplegable para selecionar las materias en las que este interesado el alumno o mas especificamente grupos si hay uno en concreto que quiera tomar por encima de los otros ofertados por la institución.

\begin{figure}[H]
    \centering
    \includegraphics[width=0.75\linewidth]{img/UImenu1.png}
    \caption{Menu de selección de materias, grupos}
    \label{fig:enter-label2}
\end{figure}

Una segunda sección del menu para indicar que prioridad tiene el estudiante respecto a evitar choques entre sus materias tomadas en la gestion o los puentes de periodos libres entre estas para evitar perder mucho tiempo.

\begin{figure}[H]
    \centering
    \includegraphics[width=0.75\linewidth]{img/UImenu2.png}
    \caption{Menu de selección de prioridad de problema}
    \label{fig:enter-label3}
\end{figure}

Una vez seleccionado las preferencias del estudiante puede mandar a generar un horarios tomando en cuenta dichas preferencia y vizualizar el resultado en una tabla como se muestra a en la imagen a continuación.

\begin{figure}[H]
    \centering
    \includegraphics[width=0.75\linewidth]{img/UIschedule.png}
    \caption{Tabla de visualización de horario}
    \label{fig:enter-label4}
\end{figure}
	
\chapter{Conclusiones y Recomendaciones}
\section{Subtitulo}
Lorem ipsum dolor sit amet 
\subsection{Subsubtitulo}
Lorem ipsum dolor sit amet 

	
	\backmatter
	
%\chapter*{Bibliografía}
\addcontentsline{toc}{chapter}{Bibliografía}
\bibliographystyle{apalike} 
%\bibliographystyle{apacite}
\bibliography{Bibliografia}

	\appendix
	\begin{appendices}
		\include{anexos}
	\end{appendices}
\end{document}