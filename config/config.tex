
%formato del texto "capitulo"
\titleformat{\chapter}[display]
{\fontsize{14}{16}\Large\filcenter} 
{{\chaptertitlename} \Roman{chapter}} %  
{20pt}  % Espacio entre "Capítulo I" y el título
{\bfseries\MakeUppercase} 

% Formato de títulos y subtítulos
\titleformat{\section}
  {\fontsize{12}{14}\bfseries\uppercase} % Estilo del título de sección
  {\thesection} % Etiqueta del capítulo
  {1em} % Espacio horizontal entre la etiqueta y el título
  {} % Código previo al título de sección

\titleformat{\subsection}{\bfseries}{\thesubsection}{1em}{}
\titleformat{\subsubsection}{\bfseries}{\thesubsubsection}{1em}{}

%formato indice
\renewcommand{\cftchapfont}{\bfseries\uppercase} %capitulos
\renewcommand{\cftsecfont}{\uppercase} % Secciones


% agrega puntos ... a los capitulos preliminares en el indice
%\renewcommand{\cftchapdotsep}{\cftdotsep}

% Formato de párrafo
\setlength{\parindent}{0pt} % Sin sangría
\setlength{\parskip}{6pt} % Espaciado entre párrafos

% Configuración de página
\pagestyle{fancy} % Estilo de página fancy
\fancyhf{} % Borra todos los encabezados y pies de página actuales
\fancyfoot[C]{\thepage} % Número de página centrado en el pie de página
\renewcommand{\headrulewidth}{0pt} % Sin línea de encabezado
\renewcommand{\footrulewidth}{0pt} % Sin línea de pie de página


% Formato de figuras
\captionsetup{justification=centering, font={footnotesize, sf}, labelfont={bf, sf}}


% Configuracion del caption
\captionsetup[table]{skip=4mm}
