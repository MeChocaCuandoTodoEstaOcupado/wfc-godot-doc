
\color{black}
%formato del texto "capitulo"
\titleformat{\chapter}[display]
{\bfseries\centering\fontsize{14}{16}} % Arial 14 negrita, centrado
{\MakeUppercase{\chaptertitlename} \ \Roman{chapter}} % CAPÍTULO I
{3pt} % Espaciado entre "Capítulo I" y el título
{\fontsize{18}{22}\selectfont\MakeUppercase} % Título también en mayúsculas sostenidas
\titlespacing*{\chapter}{0pt}{6pt}{6pt} % Espaciado antes y después


% Formato de títulos y subtítulos
\titleformat{\section}[hang]
{\fontsize{12}{14}\bfseries\MakeUppercase}
{\thesection.}
{1em}
{} 

\titleformat{\subsection}[hang]
{\fontsize{12}{14}\bfseries}
{\thesubsection.}
{1em}
{}

\titleformat{\subsubsection}{\bfseries}{\thesubsubsection}{1em}{}
%%%%%%%%%%%%%%%%%%%%%%%%%%%%%%%
% formato indice
%uppercase tiene conflictos,
%este par de lineas proboca el error en main.tex en -> \tableofcontents
%  descomentar para que en el indice se vea capitulos y secciones en mayusculas

%\renewcommand{\cftchapfont}{\bfseries\uppercase} %capitulos
%\renewcommand{\cftsecfont}{\uppercase} % Secciones
%%%%%%%%%%%%%%%%%%%%%%%%%%%%%%%

% agrega puntos ... a los capitulos preliminares en el indice
%\renewcommand{\cftchapdotsep}{\cftdotsep}

% Formato de párrafo
\setlength{\parindent}{0pt} % Sin sangría
\setlength{\parskip}{6pt} % Espaciado entre párrafos
\singlespacing % interlineado sencillo


% Configuración de página
\pagestyle{fancy} % Estilo de página fancy
\fancyhf{} % Borra todos los encabezados y pies de página actuales
\fancyfoot[C]{\thepage} % Número de página centrado en el pie de página
\renewcommand{\headrulewidth}{0pt} % Sin línea de encabezado
\renewcommand{\footrulewidth}{0pt} % Sin línea de pie de página


% Formato de figuras

\captionsetup[figure]{justification=raggedright,singlelinecheck=false}
% Configuración de la alineación de los títulos a la izquierda
\captionsetup[table]{justification=raggedright, singlelinecheck=false}
%\renewcommand{\tablename}{}
\renewcommand{\arraystretch}{1.2} % Aumentar el espacio entre las filas de las tablas

% Configuracion del caption
%\captionsetup[table]{skip=4mm}

\captionsetup{
	labelfont=bf,
	justification=raggedright, % Alinea el texto a la izquierda
	singlelinecheck=false,
	font=small
}

% Para agregar Bookmarks de imágenes y tablas
\makeatletter
\newcommand{\listoffiguresbookmarks}{%
	\pdfbookmark[0]{\listfigurename}{listoffiguresbookmark}
	\bookmarksetup{level=1}
	\@starttoc{lofb}
}
\newcommand{\listoftablesbookmarks}{%
	\pdfbookmark[0]{\listtablename}{listoftablesbookmark}
	\bookmarksetup{level=1}
	\@starttoc{lotb}
}
\makeatother