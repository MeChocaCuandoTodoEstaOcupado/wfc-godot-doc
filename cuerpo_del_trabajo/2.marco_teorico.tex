
\chapter{MARCO TEÓRICO}

En este capítulo se explican los conceptos de los mapas estilo mosaico y las técnicas a usar para poder generarlos procedural-mente, también se profundizara en la tecnología de los motores de videojuegos a usar
 \footnote{footnote: ejemplo } 
.


\section{MAPAS ESTILO MOSAICO O TILEMAPS}

Los mapas estilo mosaico o también llamados tilemaps son una técnica común en el desarrollo de videojuegos especialmente en juegos 2D, que consiste en construir el mapa del mundo o nivel de juego a base de pequeñas imágenes con forma usualmente cuadrada a los que se les llaman mosaicos o tiles,
los beneficios de usar esto es que no necesitar grandes imágenes que pueden pesar mucho en cambio se construye usando pequeñas imágenes que se pueden repetir varias vecen el diferentes partes del mapa.

Otro beneficio de esto es que se pueden poner en matrices de 2 y 3 dimensiones lo que hace sencillo poder definir las posiciones de los tiles usando algoritmos.

En el caso de mapas 3D se usan modelos 3D  o también llamados 3D mesh, en vez de imágenes  obteniendo los mismos beneficios.

\begin{figure}[H]
	\centering
	\includegraphics[width=0.6\textwidth]{img/Tilemap.jpg}
	\caption{herramienta de tilemap de Unity}
	\label{fig:tilemap}
\end{figure}

cada grilla de la matriz del mapa pudiera definirse de forma manual pero esta tarea se volvería muy pesada rápidamente en mapas grandes, por ello se busca implementar una forma de definir la matriz de forma procedural.

\section{SCP O PROBLEMAS DE SATISFACCIÓN DE RESTRICCIONES}

Para generar un mapa como este es importante que el resultado tenga coherencia, eso significa que hay grillas que si tienen cierto valor deberían de estar rodeadas solo de otros valores coherentes.

Dado lo anterior mencionado generar estos resultado entra en la categoría de problemas de satisfacción de restricciones, estos se definen como problemas matemáticos definidos como un conjunto de objetos tal que su estado debe satisfacer un número de restricciones o limitaciones.

\section{Algoritmo AC-3}

AC-3 o Arc Consistency 3, es un algoritmo conocido para la solución de problemas de satisfacción de restricciones, es la tercera iteración de esta familia de algoritmos, este en particular desarrollado por Alan Mackworth en 1977

\begin{itemize}
	\item Se empieza definiendo una lista con 2 valores por cada arista.
	\item Mientras la lista no este vaciá:
	\begin{itemize}
		\item remover una arista A a B.
		\item Por cada valor del dominio de B:
		\begin{itemize}
			\item Busca un apoyo a ese valor dada la restricción de la arista 
			\item Si ningún apoyo es encontrado:
			\begin{itemize}
			\item Remover el valor del dominio de B
			\item Añadir aristas a la lista de B a cada otra variable con la que contenga una restricción aparte de A
			\end{itemize}
		\end{itemize}
	\end{itemize} 
\end{itemize}



\section{WAVE FUNCTION COLLAPSE}

El colapso de la función de onda o wave function collapse es una implementación basada en AC-3 pero inspirado en un concepto de física cuántica con el mismo nombre. Cuando se mide un sistema cuando en superposición de estados su función de onda se colapsa a un solo estado, por lo que se considera el estado es indeterminado hasta que se haga una medición, como referencia un experimento conocidos de este fenómeno es el experimento de la doble rendija.

\begin{figure}[H]
	\centering
	\includegraphics[width=0.6\textwidth]{img/screenshot-2025-10-26_10-24-06.png}
	\caption{experimento de la doble rendija.}
	\label{fig:doble-rendija}
\end{figure}

La implementación de wave function collapse se usa principalmente para la generación procedural de bitmaps usando un bitmap de ejemplo como entrada

\begin{figure}[H]
	\centering
	\includegraphics[width=0.6\textwidth]{img/wfc.png}
	\caption{wave function collapse ejemplo.}
	\label{fig:wfc-ejemplo}
\end{figure}




\subsection{ejemplo subseccion}

\section{MOTORES DE VIDEOJUEGOS}

Usar motores de videojuegos como base de un proyecto es actualmente lo mas frecuente en el desarrollo de videojuegos ya que bastante costoso implementar desde 0 todas las funciones que aportan estos.

Por lo que los Assets necesitan ser compatibles con el motor de videojuegos que se elija usar entre ellos las opciones mas conocidas y completas serian Unity y Unreal

Unity y Unreal son motores de videojuegos los cuales han estado dominando la industria por un buen tiempo por lo cual tienen un buen repertorio de Assets publicados dando varias opciones a nuevos desarrolladores, aparte de estos hay varios otros motores entre los cuales uno adquirió popularidad recientemente Godot.

Godot es un motor de videojuegos gratuito de código abierto creado originalmente en Argentina por Ariel Manzur y Juan Linietsky como un proyecto cerrado el cual pasaría a lanzarse como código abierto el 14 de enero de 2014 con licencia MIT
Aunque adquirió popularidad rápidamente el aporte de la comunidad es bastante pequeño comparado con otros motores que llevan mas tiempo encabezando el mercado.

Una gran ventaja de usar Godot por encima de motores como Unity y Unreal es que es completamente gratuito por lo que desarrolladores no necesitan pagar licencias para publicar sus juegos, también que al ser código abierto no esta sujeto a las políticas de ninguna compañía por lo que se puede tener mas libertad creativa.

Otra ventaja de Godot es que al poder usar el mismo lenguaje de programación C\# usado en Unity y Unreal estos proyectos serian mas fáciles de exportarse a este nuevo motor de ser necesario y viceversa.

\section{ESTADO DEL ARTE}
\section{TECNOLOGÍAS}
\subsection{Lenguaje de programación}

C\#

\subsection{Godot}

Godot

\subsection{}



\subsection{GIT-GITHUB}



\section{PROCESO DE DESARROLLO}
Como se mencionó en el anterior capítulo este proyecto estará desarrollado usando el proceso ágil conocido como Kanban. en los siguientes puntos se profundizará más respecto al proceso y el motivo de su elección
\subsection{Kanban}
\subsection{Por qué no otros procesos}
