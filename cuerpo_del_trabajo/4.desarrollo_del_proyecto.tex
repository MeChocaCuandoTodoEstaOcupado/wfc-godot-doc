\chapter{DESARROLLO DEL PROYECTO}
\section{DISEÑO DE ARQUITECTURA}
Se opta por una arquitectura MVP(modelo vista presentador) para mantener desacoplada la interfaz de usuario y tener un API de uso genérico que puede ser reutilizado por otros proyectos.
Spring basa su arquitectura en MVC(Modelo vista controlador) y es la arquitectura que seguiría el componente de backend en este proyecto pero, al integrar también una interfaz de usuario ajena a Spring boot el controlador desde el punto de vista general del proyecto pasaría a usarse como un intermediario funcionando como un presentador que se encarga de la lógica y enviar o recibir información de la  vista a petición.
\begin{figure}[H]
    \centering
    \includegraphics[width=0.75\linewidth]{img/componentDiagram.png}
    \caption{Diagrama de componentes}
    \label{fig:enter-label}
\end{figure}
\section{DISEÑO DE ALGORITMO GENÉTICO}
Tras el análisis hecho a las variaciones mencionadas en el capítulo se decidió por usar la siguiente estructura [TODO].algoritmo genético simple.[TODO] analizar matemáticamente las probabilidades y cantidades máximas de combinaciones, permutaciones, relevancia en complejidad del problema, como son 6 max de materias tomadas posiblemente se utilice el algoritmo genético base pero cada parte también tiene formas de implementar revisar cuales estan disponible en Jenetics..[TODO].. dado ya el algoritmo a seguir, para el desarrollo de esto se utilizó como apoyo Jenetics una librería para el lenguaje de programación Java que cuenta con bases para la implementación de algoritmos genéticos que ayudan a agilizar la implementación de la aplicación
\section{DISEÑO DE BASE DE DATOS}
Para implementar un sistema capaz de recomendar horarios a los estudiantes, primero se necesita la información con la que trabajar, en vez de tediosamente re ingresar todos los datos para cada uso es más conveniente poder almacenar la información de los horarios en una base de datos a la cual el sistema tenga acceso para poder revisar en cada request de algún usuario para poder generar un horario óptimo.
Dado lo anterior dicho se trabajó en un diseño de la información a guardar respecto a los horarios que incluye tanto los tiempos de ingreso de las materias y de finalización, como también información básica de los docentes que darán tales materias y los grupos disponibles que estarán en la gestión.

\begin{figure}[H]
    \centering
    \includegraphics[width=0.5\linewidth]{img/db.png}
    \caption{Diagrama entidad relación}
    \label{fig:enter-label1}
\end{figure}

En la anterior imagen se puede visualizar la relación entre los datos necesario para definición de horarios

\section{DISEÑO DE INTERFAZ DE USUARIO}

En la interfaz de usuario se decidio optar por una vista simple que sea capaz de mostrar el horario resultado de las preferencias del estudiante y un menu para seleccion dichas preferecia.
\\\\
Un menu desplegable para selecionar las materias en las que este interesado el alumno o mas especificamente grupos si hay uno en concreto que quiera tomar por encima de los otros ofertados por la institución.

\begin{figure}[H]
    \centering
    \includegraphics[width=0.75\linewidth]{img/UImenu1.png}
    \caption{Menu de selección de materias, grupos}
    \label{fig:enter-label2}
\end{figure}

Una segunda sección del menu para indicar que prioridad tiene el estudiante respecto a evitar choques entre sus materias tomadas en la gestion o los puentes de periodos libres entre estas para evitar perder mucho tiempo.

\begin{figure}[H]
    \centering
    \includegraphics[width=0.75\linewidth]{img/UImenu2.png}
    \caption{Menu de selección de prioridad de problema}
    \label{fig:enter-label3}
\end{figure}

Una vez seleccionado las preferencias del estudiante puede mandar a generar un horarios tomando en cuenta dichas preferencia y vizualizar el resultado en una tabla como se muestra a en la imagen a continuación.

\begin{figure}[H]
    \centering
    \includegraphics[width=0.75\linewidth]{img/UIschedule.png}
    \caption{Tabla de visualización de horario}
    \label{fig:enter-label4}
\end{figure}